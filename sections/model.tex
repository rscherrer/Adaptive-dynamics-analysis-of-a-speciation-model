To disentangle the effect of the genetic architecture from the effect of ecology on our simulations, we derived an approximation of our model where only natural and sexual selection are at play. In this model, we keep the genetic architecture minimal. Only the ecological trait $x$ evolves, encoded by a single haploid gene and a continuum of alleles. We assume that sexual selection and assortative mating are imposed by fixed parameters. This simplification allows us to analyze the deterministic effect of selection using adaptive dynamics theory (Geritz, Metz).\\

Adaptive dynamics theory is concerned with the conditions under which an initially rare mutant can invade a population of resident phenotypes. If the geometric growth rate, or invasion fitness, of the mutant is higher than that of the resident, the mutant can invade, otherwise it disappears. Assuming small and rare mutational steps and a large and monomorphic population with respect to trait $x$, we can derive a selection gradient, which indicates the direction of evolution. We find evolutionary equilibria as values of $x$ where the selection gradient becomes zero. Speciation is favored if selection leads to those equilibria and favors branching, that is, becomes disruptive once there. Those conditions can be evaluated by studying the attainability and the invasion-proofness of the equilibria, using criteria that can be derived from the selection gradient and the curvature of the fitness function. We are searching for branching points, i.e. equilibrium values of trait $x$ towards which directional selection leads but where the fitness landscape becomes a valley once attained, thus invasible by any neighboring mutant. It is important to note that the shape of the fitness landscape can change as a population evolves, due to frequency-dependent selection. Here, frequency-dependence is present through competition for limited resources.\\

We first describe the dynamics of a mutant population and how they depend on model parameters. We derive the invasion fitness as the geometric growth rate of the mutant. Next, we derive an expression for the selection gradient, the curvature of the fitness function and the attainability of the equilibriq. We show that the location of evolutionary equilibria does not depend on sexual selection, but that sexual selection in the form of assortative mating has a stabilizing effect on the evolution of trait $x$ and may counteract branching. Finally, we explore branching conditions across parameter space by numerically evaluating the previously derived functions, using a \textit{Mathematica} notebook provided in Online Supplements.

\subsection*{Ecological dynamics}

Consider a population of organisms with access to two habitat patches and two resources, just like in our simulation model. The resource dynamics are affected by the consumption by the organisms as in Equation \ref{eq:resource_dynamics}, except that here we assume the population to be monomorphic with respect to the ecological trait $x$. Equation \ref{eq:resource_dynamics} then becomes

\begin{equation}
    \frac{dR_{ij}}{dt} = a \, H_{ij} - (b + e_i(x)\, N_j ) \, R_{ij}
\end{equation}

where $R_{ij}$ is the concentration of resource $i$ in patch $j$ and depends on the number of individuals $N_j$ in patch $j$. The other parameters are described in the main text. The attack rates $e_i(x)$ are given in Equation \ref{eq:attack_rates} and are subject to a trade-off controlled by the ecological selection coefficient $s$ (Fig. \ref{fig:attackrates}).\\

The population dynamics of a rare mutant with ecological trait value $x$ arising in a monomorphic resident population with trait value $\hat{x}$ are

\begin{equation}
    \overrightarrow{N'}_{t+1} = \pmb{\Lambda} \, \overrightarrow{N'}_t
\end{equation}

where $\overrightarrow{N'}_t$ is the vector of mutant population densities across patches, $N'_1$ and $N'_2$, at discrete generation $t$. The transition matrix $\pmb{\Lambda}$ captures the steps of the life cycle of the organism,

\begin{equation}
    \pmb{\Lambda} = \pmb{M} \, \pmb{Q}
\end{equation}

where $\pmb{M}$ is the migration matrix, given by

\begin{equation}
    \pmb{M} =
    \begin{pmatrix}
        1-m & m \\
        m & 1-m 
    \end{pmatrix}
\end{equation}

where $m$ is the migration rate, and $\pmb{Q}$ is the matrix of local per capita growth rates,

\begin{equation}
    \pmb{Q} =
    \begin{pmatrix}
        r_1(x, \hat{x}) & 0 \\
        0 & r_2(x, \hat{x})
    \end{pmatrix}
\end{equation}

The within-patch geometric growth rate captures local competition, reproduction and survival. The local growth rate in patch $j$ is given by

\begin{equation}
    r_j(x, \hat{x}) = 1 - d + \frac{1}{2} \, \Bigg[ W_j(x, \hat{x}) + W_j(\hat{x}, \hat{x}) \, \frac{W_j(x, \hat{x})\,A(x,\hat{x})}{W_j(\hat{x}, \hat{x})\,A(\hat{x},\hat{x})}\Bigg]
    \label{eq:growth_rate}
\end{equation}

where $d$ is the per capita death rate, $W_j(x,\hat{x})$ is the reproductive success of a mutant with trait value $x$ in a resident population with trait $\hat{x}$ in patch $j$, and $A(x, \hat{x})$ is the attractiveness of a male mutant for a resident female.\\

The term $1-d$ represents the survival of adults from one generation to the next. The rest of the equation represents the production of mutant offspring by both males and female, respectively. In females, the reproductive success $W(x, \hat{x})$ directly translates into number of offspring. Males, on the other hand, do not produce offspring directly. Given that the mutant is rare, we assume that mutants only mate with residents, and a mutant male will therefore only contribute to the offspring of a resident female, $W(\hat{x}, \hat{x})$. Males are also subject to sexual selection. Their reproductive output is weighed by their mating success. Male mating success depends on the attractiveness of the male for a resident female, $A(x, \hat{x})$, and on the reproductive success of the male $W(x, \hat{x})$, which in males determines their activity on the mating market. The mating success of the mutant is normalized by the mating success of the resident. Finally, in a haploid system with Mendelian segregation, only half of the offspring of mutants mating with residents will be of mutant phenotype.\\

The reproductive success is the competitive ability of the individual when exploiting the resources,

\begin{equation}
    W_j(x, \hat{x}) = e_1(x) \, R^*_{1j} + e_2(x) \, R^*_{2j}
\end{equation}

where $R^*_{ij}$ is the equilibrium concentration of resource $i$ in patch $j$. Just like in our simulation model we assume fast resource dynamics relative to the population of organisms. We also assume that mutants are rare enough that they have no effect on the equilibrium resource dynamics, such that

\begin{equation}
    R^*_{ij} = \frac{a \, H_{ij}}{b + e_i(\hat{x})\,N_j}
\end{equation}

The attractiveness function is the same as the assortative mating function in Equation \ref{eq:assortative_mating}, except that the female mate preference trait $y$ is absent. This is because we are interested in the branching conditions under various degrees of assortative mating and not in the evolution of assortative mating per se. This is also why we do not allow for disassortative mating here. Assortative mating is captured by the sexual selection coefficient $\alpha$ in

\begin{equation}
    A(x, \hat{x}) = \exp{(-\alpha \, (x - \hat{x})^2)}
\end{equation}

Given that $A(\hat{x}, \hat{x}) = 1$, Equation \ref{eq:growth_rate} simplifies to

\begin{equation}
    r_j(x, \hat{x}) = 1 - d + \frac{1}{2} \, W(x, \hat{x}) \, \big(1 + A(x, \hat{x})\big)
\end{equation}

In the following we use the convention $r_j(x, \hat{x}) = r_j$ for readability. The invasion fitness, i.e. the long-term growth rate of the mutant, is the leading eigenvalue of the transition matrix $\pmb{\Lambda}$,

\begin{equation}
    \lambda(x, \hat{x}) = \frac{1}{2} \bigg[(1-m) \, (r_1 + r_2) + \sqrt{(1-m)^2 \, (r_1^2 - 2 \, r_1 \, r_2 + r_2^2) + 4 \, m^2 \, r_1 \, r_2} \bigg]
    \label{eq:invasion_fitness}
\end{equation}

where $\lambda$ is the growth rate of a mutant arising in a resident population at its demographic equilibrium, as defined by trait value $\hat{x}$. The mutant can invade and become the new resident if $\lambda(x, \hat{x}) > 1$, otherwise the mutant goes extinct. Note that $\lambda(\hat{x}, \hat{x}) = 1$. In the following we use the notation $\lambda$ for $\lambda(x, \hat{x})$.

\subsection*{Adaptive dynamics}

\paragraph{Selection gradient} The direction of evolution by selection, assuming small mutational steps and a very large population, is given by the selection gradient,

\begin{equation}
    G(\hat{x}) = \frac{\partial \lambda}{\partial x}\bigg|_{x=\hat{x}}
\end{equation}

which is the derivative of the invasion fitness with respect to the mutant phenotype, evaluated at the resident phenotype. Differentiating Equation \ref{eq:invasion_fitness} is possible but cumbersome, hence we find an expression for the selection gradient using the rule:

\begin{equation}
   \frac{\partial \lambda}{\partial x}\bigg = \frac{1}{\overrightarrow{v}^T\,\overrightarrow{u}} \, \bigg( \overrightarrow{v}^T \, \frac{\partial \pmb{\Lambda}}{\partial x} \, \overrightarrow{u} \bigg)
   \label{eq:deriv_fitness}
\end{equation}

(citation) where $\overrightarrow{v}$ and $\overrightarrow{u}$ are left and right eigenvectors, respectively, of the transition matrix $\pmb{\Lambda}$ associated with the dominant eigenvalue $\lambda$. Possible such eigenvectors are

\begin{equation}
    \overrightarrow{v} = 
    \begin{pmatrix}
        \lambda - (1-m)\,r_2 \\
        m\,r_2 
    \end{pmatrix}
    \label{eq:left_eigenvector}
\end{equation}

and

\begin{equation}
    \overrightarrow{u} = 
    \begin{pmatrix}
        \lambda - (1-m)\,r_2 \\
        m\,r_1
    \end{pmatrix}
\end{equation}

The derivative of the transition matrix $\pmb\Lambda$ with respect to $x$ is

\begin{equation}
    \frac{\partial \pmb{\Lambda}}{\partial x} = \pmb{M} \, \frac{\partial \pmb{Q}}{\partial x}
\end{equation}

where $\partial \pmb{Q} / \partial x$ is a diagonal matrix whose elements are, given for patch $j$,

\begin{equation}
    \frac{\partial r_j}{\partial x} = \frac{1}{2} \, \bigg[\frac{\partial W_j}{\partial x} \, (1 + A(x, \hat{x})) + \frac{\partial A}{\partial x} \, W_j(x, \hat{x})\bigg]
    \label{eq:deriv_local_growth}
\end{equation}

where

\begin{equation}
    \frac{\partial A}{\partial x} = -2 \, \alpha \, (x - \hat{x}) \, A(x, \hat{x})
\end{equation}

Because the selection gradient is evaluated at the resident trait value, where $\partial A / \partial x = 0$ and $A(\hat{x},\hat{x}) = 1$, Equation \ref{eq:deriv_local_growth} reduces to

\begin{equation}
    \frac{\partial r_j}{\partial x}\bigg|_{x=\hat{x}} = \frac{\partial W_j}{\partial x}\bigg|_{x=\hat{x}} 
\end{equation}

Also, given $\lambda(\hat{x}, \hat{x}) = 1$, the eigenvectors $\overrightarrow{v}$ and $\overrightarrow{u}$ evaluated at the resident simplify to

\begin{equation}
    \hat{\overrightarrow{v}} = 
    \begin{pmatrix}
        1 - (1-m)\hat{r}_2\\
        m \, \hat{r}_2
    \end{pmatrix}
    \label{eq:left_eigenvector_resident}
\end{equation}

and

\begin{equation}
    \hat{\overrightarrow{u}} = 
    \begin{pmatrix}
        1 - (1-m)\hat{r}_2\\
        m \, \hat{r}_1
    \end{pmatrix}
    \label{eq:right_eigenvector_resident}
\end{equation}

where 

\begin{equation}
    \hat{r}_j = r_j(\hat{x}, \hat{x}) = 1 - d + W_j(\hat{x}, \hat{x})
    \label{eq:local_growth_resident}
\end{equation}

Substituting Equations \ref{eq:left_eigenvector}--\ref{eq:local_growth_resident} into Equation \ref{eq:deriv_fitness} and simplifying, we get the following expression for the selection gradient:

\begin{equation}
    G(\hat{x}) = \frac{\big[1-m-(2-4\,m+m^2)\,\hat{r}_2+(1-3\,m+2\,m^2)\,\hat{r}^2_2\big]\, \hat{W}_1' + m^2\,\hat{r}_1\,\hat{W}_2'}{(1-\hat{r}_2+m\,\hat{r}_2)^2 + m^2\,\hat{r}_1\,\hat{r}_2}
    \label{eq:fitness_gradient_expression}
\end{equation}

where $\hat{W}_j' = \partial W_j / \partial x |_{x = \hat{x}}$.\\

Hence, the selection gradient does not involve the sexual selection coefficient $\alpha$, and is identical to the selection gradient of an asexual model (show this), whose local growth rate is given by Equation \ref{eq:local_growth_resident}. This means that the equilibrium values of ecological trait $x$, those where the selection gradient is zero, depend on competition, survival and migration, but not on sexual selection. We will now show that sexual selection affects the stability of these equilibria.

\paragraph{Evolutionary stability} The evolutionary stability of an equilibrium trait value $x^*$ depends on the second derivative, or curvature, of the invasion fitness with respect to the mutant phenotype, evaluated at the resident phenotype at evolutionary equilibrium:

\begin{equation}
    \frac{\partial^2 \lambda}{\partial x^2}\bigg|_{x=\hat{x}=x^*}
\end{equation}

If this value is positive, the equilibrium is evolutionarily unstable and the trait value is a fitness minimum, which is a requirement for branching. If this value is negative, the equilibrium is an invasion-proof local maximum, i.e. an evolutionarily stable strategy (ESS) and it cannot branch. It follows from Equation \ref{eq:deriv_fitness} that

\begin{equation}
    \frac{\partial^2 \lambda}{\partial x^2} = \frac{\overrightarrow{v}^T\,\overrightarrow{u}\,\frac{\partial}{\partial x}\,\big(\overrightarrow{v}^T\,\frac{\partial \pmb{\Lambda}}{\partial x}\,\overrightarrow{u}\big) - \frac{\partial}{\partial x} \big( \overrightarrow{v}^T \, \overrightarrow{u} \big) \, \overrightarrow{v}^T \, \frac{\partial \pmb{\Lambda}}{\partial x}\,\overrightarrow{u}}{(\overrightarrow{v}^T\,\overrightarrow{u})^2}
\end{equation}

which, because

\begin{equation}
    \bigg( \overrightarrow{v}^T\,\frac{
    \partial \pmb \Lambda}{\partial x}\,\overrightarrow{u} \bigg)_* = G(x^*) = 0
\end{equation}

simplifies, at equilibrium $x^*$, to

\begin{equation}
    \frac{\partial^2 \lambda}{\partial x^2}\bigg|_* = \Bigg[ \frac{1}{\overrightarrow{v}^T\,\overrightarrow{u}}\,\frac{\partial}{\partial x}\,\bigg(\overrightarrow{v}^T\,\frac{\partial \pmb{\Lambda}}{\partial x}\,\overrightarrow{u}\bigg) \Bigg]_*
    \label{eq:fitness_curvature_equilibrium}
\end{equation}

where

\begin{equation}
    \frac{\partial}{\partial x} \, \bigg(\overrightarrow{v}^T \, \frac{\partial \pmb{\Lambda}}{\partial x} \, \overrightarrow{u}\bigg) = \frac{\partial \overrightarrow{v}^T}{\partial x}\,\frac{\partial \pmb{\Lambda}}{\partial x}\,\overrightarrow{u} + \overrightarrow{v}^T\,\frac{\partial^2 \pmb{\Lambda}}{\partial x^2}\,\overrightarrow{u} + \overrightarrow{v}^T\,\frac{\partial \pmb{\Lambda}}{\partial x}\,\frac{\partial \overrightarrow{u}}{\partial x}
    \label{eq:deriv_num_gradient_unevaluated}
\end{equation}

where the derivatives of the left and right eigenvectors are given by

\begin{equation}
    \frac{\partial \overrightarrow{v}}{\partial x} = 
    \begin{pmatrix}
        \nicefrac{\partial \lambda}{\partial x} - (1-m)\,\nicefrac{\partial r_2}{\partial x}\\
        m\,\nicefrac{\partial r_2}{\partial x}
    \end{pmatrix}
\end{equation}

and

\begin{equation}
    \frac{\partial \overrightarrow{u}}{\partial x} = 
    \begin{pmatrix}
        \nicefrac{\partial \lambda}{\partial x} - (1-m)\,\nicefrac{\partial r_2}{\partial x}\\
        m\,\nicefrac{\partial r_1}{\partial x}
    \end{pmatrix}
\end{equation}

which, because $\partial r_j / \partial x = \partial W_j / \partial x$ at the resident $\hat x$ and $\partial \lambda / \partial x = 0$ at the equilibrium $x^*$, respectively evaluate to

\begin{equation}
    \frac{\partial \overrightarrow{v}}{\partial x}\bigg|_{x=\hat x=x^*} = 
    \begin{pmatrix}
        (m-1)\,\nicefrac{\partial W_2}{\partial x}\\
        m\,\nicefrac{\partial W_2}{\partial x}
    \end{pmatrix}
    _{x=\hat x=x^*}
\end{equation}

and

\begin{equation}
    \frac{\partial \overrightarrow{u}}{\partial x}\bigg|_{x=\hat x=x^*} = 
    \begin{pmatrix}
        (m-1)\,\nicefrac{\partial W_2}{\partial x}\\
        m\,\nicefrac{\partial W_1}{\partial x}
    \end{pmatrix}
    _{x=\hat x=x^*}
    \label{eq:right_eigenvector_equilibrium}
\end{equation}

and where the second derivative of the transition matrix is given by 

\begin{equation}
    \frac{\partial^2 \pmb{\Lambda}}{\partial x^2} = \pmb{M} \, \frac{\partial^2 \pmb{Q}}{\partial x^2}
    \label{eq:second_derivative_transition}
\end{equation}

where $\partial^2 \pmb{Q} / \partial x^2$ is a diagonal matrix whose elements for each patch $j$ are given by differentiating Equation \ref{eq:deriv_local_growth}:

\begin{equation}
    \frac{\partial^2 r_j}{\partial x^2} = \frac{1}{2} \bigg[\frac{\partial^2 W_j}{\partial x^2}\,\big(1+A(x,\hat{x})\big) + 2\,\frac{\partial W_j}{\partial x}\,\frac{\partial A}{\partial x} + \frac{\partial^2 A}{\partial x^2}\,W_j(x,\hat{x})\bigg]
\end{equation}

where

\begin{equation}
    \frac{\partial^2 A}{\partial x^2} = -2\,\alpha\,\big( A(x,\hat{x}) + (x-\hat{x})\,\nicefrac{\partial A}{\partial x} \big)
\end{equation}

which, at equilibrium $x=\hat{x}=x^*$ becomes

\begin{equation}
    \frac{\partial^2 A}{\partial x^2}\bigg|_{x=\hat{x}=x^*} = -2\,\alpha
\end{equation}

The second derivatives of the local growth rates therefore simplify to

\begin{equation}
    \frac{\partial^2 r_j}{\partial x^2}\bigg|_{x=\hat{x}=x^*}= \frac{\partial^2 W_j}{\partial x^2}\bigg|_{x=\hat{x}=x^*} - \alpha \, W_j(x^*,x^*)
    \label{eq:second_derivative_local_growth_equilibrium}
\end{equation}

By substituting Equations \ref{eq:left_eigenvector_resident}, \ref{eq:right_eigenvector_resident} and \ref{eq:deriv_num_gradient_unevaluated}--\ref{eq:second_derivative_local_growth_equilibrium} into Equation \ref{eq:fitness_curvature_equilibrium} and simplifying, we obtain an expression for the curvature of the fitness function at equilibrium:

\begin{equation}
    \begin{split}
        \frac{\partial^2 \lambda}{\partial x^2}\bigg|_{x=\hat x=x^*}=&\frac{1}{1 + m^2\,r^*_1\,r^*_2 + (m - 1)\,\big(2 + (m - 1)\,r^*_2^2\big)} \times \\
        &\Bigg[m^2\, r^*_ 1\, W''^*_2 + 2\, (2\, m - 1)\, \big(1 + (m - 1)\, r^*_ 2\big)\, W'^*_1 \, W'^*_2 +\\
        &\big(1 + (m - 1) \, r^*_ 2\big)\, \big(1 - m + (2 m - 1) \,r^*_ 2\big)\, W''^*_1 -\\
        &\alpha \times \Big(m^2\, r^*_ 1\, W^*_2 - \\
        & \big(m - 1 + (2 - 4\,m + m^2)\, r^*_ 2 - (1 - 3\,m + 2\,m^2)\,r^*_ 2^2\big)\,W^*_1 \Big)\Bigg]
    \end{split}
    \label{eq:fitness_curvature_equilibrium_expression}
\end{equation}

where $W_j^* = W_j(x^*, x^*)$, $W_j''^* = \partial^2 W_j / \partial x^2 |_{x=\hat{x}=x^*}$ and $r_j^* = r_j(x^*, x^*)$.\\

We used this expression to assess the stability of evolutionary equilibria. We will now show that the fitness curve gets more concave at equilibrium as the sexual selection coefficient $\alpha$ increases, i.e. that assortative mating has a stabilizing effect on evolutionary equilibria and may prevent branching. For this, assume that the local per capita growth rate $r_j$ depends on two traits, $x$, which affects competitive ability, and $y$, which affects attractiveness. Those two traits are eventually the same, but we split them to study the effect of the ecological trait on ecological and sexual selection separately. We first show that $x$ and $y$ have independent effects on the curvature of the fitness function at equilibrium, that is,

\begin{equation}
    \frac{\partial^2 \lambda}{\partial x \partial y}\bigg|_* = 0
\end{equation}

where $*$ symbolizes $x=\hat x=x^*$, $y=\hat y=y^*$. The cross-derivative of the fitness function with respect to $x$ and $y$ is

\begin{equation}
    \frac{\partial^2 \lambda}{\partial x \partial y} = \frac{\partial}{\partial y}\,\Bigg[\frac{1}{\overrightarrow{v}^T\,\overrightarrow{u}}\,\bigg( \overrightarrow{v}^T\,\frac{\partial \Lambda}{\partial x}\,\overrightarrow{u} \bigg)\Bigg]
\end{equation}

which is equivalent to

\begin{equation}
    \frac{\partial^2 \lambda}{\partial x \partial y} = \frac{\frac{\partial}{\partial y}\Big(\overrightarrow{v}^T\,\frac{\partial \pmb \Lambda}{\partial x}\,\overrightarrow{u}\Big)\,\overrightarrow{v}^T\,\overrightarrow{u}-\frac{\partial}{\partial y}\,\Big(\overrightarrow{v}^T\,\overrightarrow{u}\Big)\,\Big(\overrightarrow{v}^T\,\frac{\partial \pmb \Lambda}{\partial x}\,\overrightarrow{u}\Big)}{(\overrightarrow{v}^T\,\overrightarrow{u})^2}
\end{equation}

in which

\begin{equation}
    \bigg(\overrightarrow{v}^T\,\frac{\partial \pmb \Lambda}{\partial x}\,\overrightarrow{u}\bigg)_* = 0
\end{equation}

as it corresponds to the fitness gradient, and

\begin{equation}
    \frac{\partial}{\partial y}\,\bigg(\overrightarrow{v}^T\,\frac{\partial \pmb \Lambda}{\partial x}\,\overrightarrow{u}\bigg) = \frac{\partial \overrightarrow{v}^T}{\partial y}\,\frac{\partial \pmb \Lambda}{\partial x}\,\overrightarrow{u}+\overrightarrow{v}^T\,\frac{\partial^2 \pmb \Lambda}{\partial x \partial y}\,\overrightarrow{u}+\overrightarrow{v}^T\,\frac{\partial \pmb \Lambda}{\partial x}\,\frac{\partial \overrightarrow{u}}{\partial y}   
    \label{eq:deriv_lambda_x_y}
\end{equation}

where the derivatives of the left and right eigenvectors with respect to $y$ are, respectively,

\begin{equation}
    \frac{\partial \overrightarrow{v}}{\partial y} =
    \begin{pmatrix}
        \nicefrac{\partial \lambda}{\partial y} - (1-m)\,\nicefrac{\partial r_2}{\partial y}\\
        m \,\nicefrac{\partial r_2}{\partial y}
    \end{pmatrix}
\end{equation}

and

\begin{equation}
    \frac{\partial \overrightarrow{u}}{\partial y} =
    \begin{pmatrix}
        \nicefrac{\partial \lambda}{\partial y} - (1-m)\,\nicefrac{\partial r_2}{\partial y}\\
        m \,\nicefrac{\partial r_1}{\partial y}
    \end{pmatrix}
\end{equation}

where $\partial \lambda / \partial y = 0$ at equilibrium $x^*$, $y^*$, and

\begin{equation}
    \frac{\partial r_j}{\partial y}\bigg|_* = \frac{1}{2} \, \frac{\partial A}{\partial y}\bigg|_*\,W_j(x^*,x^*)
\end{equation}

where $\partial A / \partial y = 0$ at equilibrium $y^*$. Therefore, both $\partial \overrightarrow{v} / \partial y$ and $\partial \overrightarrow{u} / \partial y$ are vectors of zeros at equilibrium. The cross-derivative of the transition matrix is

\begin{equation}
    \frac{\partial^2 \pmb \Lambda}{\partial x \partial y} = \pmb M \, \frac{\partial}{\partial y}\,\frac{\partial \pmb Q}{\partial x}
\end{equation}

where 

\begin{equation}
    \frac{\partial \pmb Q}{\partial x} = \frac{1}{2}\,\big(1+A(y,\hat y)\big)\,\pmb M \frac{\partial \pmb W}{\partial x}
\end{equation}

where $\pmb W$ is independent of trait $y$ and so

\begin{equation}
    \frac{\partial}{\partial y}\,\frac{\partial \pmb Q}{\partial x} = \frac{1}{2}\,\frac{\partial A}{\partial y}\,\pmb M\,\frac{\partial \pmb W}{\partial x}
\end{equation}

which also reduces to zero at equilibrium and therefore, 

\begin{equation}
    \frac{\partial^2 \lambda}{\partial x \partial y}\bigg|_* = 0
\end{equation}

therefore competition and mate choice  have independent effects on the stability of equilibria. The effect of mate choice on the stability of an equilibrium is given by

\begin{equation}
    \frac{\partial^2 \lambda}{\partial y^2}\bigg|_*
\end{equation}

If this expression is negative, sexual selection is stabilizing the evolutionary dynamics of the ecological trait and may prevent branching. The curvature of the fitness function with respect to $y$ is given by

\begin{equation}
    \frac{\partial^2 \lambda}{\partial y^2} = \frac{1}{\overrightarrow{v}^T\,\overrightarrow{u}}\,\bigg(\overrightarrow{v}^T\frac{\partial \pmb \Lambda}{\partial y}\,\overrightarrow{u}\bigg)
\end{equation}

where 

\begin{equation}
    \frac{\partial \pmb \Lambda}{\partial y} = \pmb M \frac{\partial \pmb Q}{\partial y}
\end{equation}

where

\begin{equation}
    \frac{\partial \pmb Q}{\partial y} = \frac{1}{2}\,\frac{\partial A}{\partial y}\,W(x,\hat x)
\end{equation}

therefore reducing to

\begin{equation}
    \frac{\partial^2 \lambda}{\partial y^2} = \frac{1}{2}\,\frac{\partial}{\partial y}\,\bigg(\frac{\partial A}{\partial y}\, \frac{\overrightarrow{v}^T\,\pmb W\,\overrightarrow{u}}{\overrightarrow{v}^T\,\overrightarrow{u}}\bigg)
\end{equation}

which is equivalent to

\begin{equation}
    \frac{\partial^2 \lambda}{\partial y^2} = \frac{1}{2}\,\frac{\frac{\partial}{\partial y}\,\Big(\frac{\partial A}{\partial y}\,\overrightarrow{v}^T\,\pmb W\,\overrightarrow{u}\Big)\,\overrightarrow{v}^T\,\overrightarrow{u}-\frac{\partial}{\partial y}\Big(\overrightarrow{v}^T\,\overrightarrow{u}\Big)\frac{\partial A}{\partial y}\,\Big(\overrightarrow{v}^T\,\pmb W\,\overrightarrow{u}\Big)}{(\overrightarrow{v}^T\,\overrightarrow{u})^2}
\end{equation}

where $\partial A / \partial y = 0$ at equilibrium and which therefore simplifies to

\begin{equation}
    \frac{\partial^2 \lambda}{\partial y^2}\bigg|_* = \frac{1}{2}\,\Bigg(\frac{1}{\overrightarrow{v}^T\,\overrightarrow{u}}\,\frac{\partial}{\partial y}\,\bigg[\frac{\partial^2 A}{\partial y^2}\,\Big(\overrightarrow{v}^T\,\pmb W\,\overrightarrow{u}\Big)+\frac{\partial A}{\partial y}\,\frac{\partial \big(\overrightarrow{v}^T\,\pmb W\,\overrightarrow{u}\big)}{\partial y}\bigg]\Bigg)_*
\end{equation}

which further simplifies to

\begin{equation}
    \frac{\partial^2 \lambda}{\partial y^2}\bigg|_* = \frac{1}{2}\,\Bigg(\frac{\overrightarrow{v}^T\,\pmb W\,\overrightarrow{u}}{\overrightarrow{v}^T\,\overrightarrow{u}}\,\frac{\partial^2 A}{\partial y}\Bigg)_* = -\alpha \, \Bigg(\frac{\overrightarrow{v}^T\,\pmb W\,\overrightarrow{u}}{\overrightarrow{v}^T\,\overrightarrow{u}}\Bigg)_*
\end{equation}

where 

\begin{equation}
    \overrightarrow{v}^*^T\,\overrightarrow{u}^* = \big(1-(1-m)\,r^*_2\big)^2 + m^2\,r^*_1\,r^*_2 > 0
\end{equation}

assuming that the per capita growth rates $r^*_j$ have the same sign, and whose sign is therefore given by the numerator,

\begin{equation}
    \overrightarrow{v}^*^T\,\pmb W^*\,\overrightarrow{u}^* = \big(1-(1-m)\,r^*_2\big)^2\,W^*_1 + m^2\,r^*_1\,r^*_2\,W^*_2
\end{equation}

which is also positive. Therefore, the curvature of the fitness function $\partial^2 \lambda / \partial y^2$ at equilibrium depends negatively on the sexual selection coefficient $\alpha$, i.e. assortative mating brings stabilizing sexual selection into the evolutionary dynamics.

\paragraph{Convergence stability} Last, attainability, or convergence stability, of an equilibrium $x^*$ is guaranteed if the selection gradient favors increasing $x$ when $x < x^*$ but decreasing $x$ when $x > x^*$, that is

\begin{equation}
    \frac{\partial G}{\partial \hat{x}}\bigg|_{\hat{x}=x^*} < 0
\end{equation}

where 

\begin{equation}
    \frac{\partial G}{\partial \hat{x}} = \frac{\partial}{\partial \hat{x}} \Bigg[ \frac{1}{\hat{\overrightarrow{v}}^T\,\hat{\overrightarrow{u}}} \bigg( \hat{\overrightarrow{v}}^T\,\frac{\partial \pmb{\Lambda}}{\partial x}\bigg|_{x=\hat{x}}\,\hat{\overrightarrow{u}} \bigg) \Bigg]
\end{equation}

which is equivalent to

\begin{equation}
    \frac{\partial G}{\partial \hat{x}} = \frac{\frac{\partial}{\partial \hat{x}} \big( \hat{\overrightarrow{v}}^T\,\frac{\partial \pmb{\Lambda}}{\partial x}\big|_{x=\hat{x}}\,\hat{\overrightarrow{u}} \big)\,\hat{\overrightarrow{v}}^T\,\hat{\overrightarrow{u}} - \big( \hat{\overrightarrow{v}}^T\,\frac{\partial \pmb{\Lambda}}{\partial x}\big|_{x=\hat{x}}\,\hat{\overrightarrow{u}} \big)\,\frac{\partial}{\partial \hat{x}} \big(\hat{\overrightarrow{v}}^T\,\hat{\overrightarrow{u}}\big)}{(\hat{\overrightarrow{v}}^T\,\hat{\overrightarrow{u}})^2}
    \label{eq:deriv_gradient}
\end{equation}

which, because

\begin{equation}
    \bigg( \overrightarrow{v}^T\,\frac{\partial \pmb \Lambda}{\partial x}\,\overrightarrow{u} \bigg)_* = G(x^*) = 0
    \label{eq:gradient_is_zero_at_equilibrium}
\end{equation}

simplifies, at equilibrium $x^*$, to

\begin{equation}
    \frac{\partial G}{\partial \hat x}\bigg|_* = \Bigg[ \frac{1}{\hat{\overrightarrow{v}}\,\hat{\overrightarrow{u}}} \, \frac{\partial}{\partial \hat x}\,\bigg(\hat{\overrightarrow{v}}^T\,\frac{\partial \pmb \Lambda}{\partial x}\bigg|_{x=\hat x}\,\hat{\overrightarrow{u}}\bigg)\Bigg]_*
    \label{eq:deriv_gradient_equilibrium}
\end{equation}

where

\begin{equation}
    \frac{\partial}{\partial \hat{x}} \bigg( \hat{\overrightarrow{v}}^T\,\frac{\partial \pmb{\Lambda}}{\partial x}\bigg|_{x=\hat{x}}\,\hat{\overrightarrow{u}} \bigg) = \frac{\partial \hat{\overrightarrow{v}}^T}{\partial \hat{x}}\,\frac{\partial \pmb{\Lambda}}{\partial x}\bigg|_{x=\hat{x}}\,\hat{\overrightarrow{u}} + \hat{\overrightarrow{v}}^T\,\frac{\partial}{\partial \hat{x}} \bigg(\frac{\partial \pmb{\Lambda}}{\partial x}\bigg|_{x=\hat{x}}\bigg)\,\hat{\overrightarrow{u}} + \hat{\overrightarrow{v}}^T\,\frac{\partial \pmb{\Lambda}}{\partial x}\bigg|_{x=\hat{x}}\,\frac{\partial \hat{\overrightarrow{u}}}{\partial \hat{x}}
\end{equation}

where $\partial \pmb \Lambda / \partial x = \pmb M \, \partial \pmb W / \partial x$ at the resident $\hat x$ and therefore

\begin{equation}
    \frac{\partial}{\partial \hat x}\,\bigg(\frac{\partial \pmb \Lambda}{\partial x}\bigg|_{x=\hat x}\bigg) = \pmb M \, \frac{\partial}{\partial \hat x}\,\bigg(\frac{\partial \pmb W}{\partial x}\bigg|_{x=\hat x}\bigg)
\end{equation}

and where the change in left and right eigenvectors evaluated at the resident, as the resident changes, are

\begin{equation}
    \frac{\partial \hat{\overrightarrow{v}} }{\partial \hat x} = 
    \begin{pmatrix}
        (m-1)\,\nicefrac{\partial \hat r_2}{\partial \hat x}\\
        m\,\nicefrac{\partial \hat r_2}{\partial \hat x}
    \end{pmatrix}
    \label{eq:deriv_left_eigenvector_resident}
\end{equation}

and

\begin{equation}
    \frac{\partial \hat{\overrightarrow{u}} }{\partial \hat x} = 
    \begin{pmatrix}
        (m-1)\,\nicefrac{\partial \hat r_2}{\partial \hat x}\\
        m\,\nicefrac{\partial \hat r_1}{\partial \hat x}
    \end{pmatrix}
    \label{eq:deriv_right_eigenvector_resident}
\end{equation}

where $\hat r_j(\hat x) = r_j(\hat x, \hat x)$ is the local growth rate of the resident. The local growth rate is a function of the equilibrium resource concentrations $R^*_{ij}(\hat x)$, which are themselves functions of the equilibrium population densities $N^*_j(\hat x)$ in each patch. The equilibrium population densities depend in return on the equilibrium resource concentrations $R^*_{ij}(\hat x)$ and are found, for a given resident trait value $\hat x$, by solving the resident ecological dynamics:

\begin{equation}
    \pmb \Lambda(\hat x)
    \begin{pmatrix}
        N^*_1(\hat x)\\
        N^*_2(\hat x)
    \end{pmatrix}
    =
    \begin{pmatrix}
        N^*_1(\hat x)\\
        N^*_2(\hat x) 
    \end{pmatrix}
\end{equation}

which are third-degree rational equations in $N^*_1$ and $N^*_2$, which we could not solve analytically. Therefore, we could not obtain explicit expressions for the derivatives of $N^*_1$ and $N^*_2$ with respect to the resident trait value $\hat x$, which are needed to compute the derivatives of the local growth rates $\hat r_j(\hat x)$ in the derivatives of the eigenvectors in Equations \ref{eq:deriv_left_eigenvector_resident} and \ref{eq:deriv_right_eigenvector_resident}, respectively. Instead, we used differences in selection gradient close to the singular points to calculate the convergence stability criterion.

\subsection*{Numerical procedures}

We numerically explored phenotype space for evolutionary equilibria $x^*$ by solving the root of the fitness gradient given in Equation \ref{eq:fitness_gradient_expression} across a range of parameter combinations. For each equilibrium we assessed its convergence and evolutionary stability by evaluating the fitness curvature (Equation \ref{eq:fitness_curvature_equilibrium_expression}) and the change in the direction of the gradient, respectively. Convergent stable but evolutionarily unstable equilibria were classified as branching points. The procedure is shown in the accompanying \textit{Mathematica} notebook. The results of our exploration of parameter space are shown in Figure \ref{fig:adaptive_dynamics}.




